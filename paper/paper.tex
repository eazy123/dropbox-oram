
%% bare_conf.tex
%% V1.3
%% 2007/01/11
%% by Michael Shell
%% See:
%% http://www.michaelshell.org/
%% for current contact information.
%%
%% This is a skeleton file demonstrating the use of IEEEtran.cls
%% (requires IEEEtran.cls version 1.7 or later) with an IEEE conference paper.
%%
%% Support sites:
%% http://www.michaelshell.org/tex/ieeetran/
%% http://www.ctan.org/tex-archive/macros/latex/contrib/IEEEtran/
%% and
%% http://www.ieee.org/

%%*************************************************************************
%% Legal Notice:
%% This code is offered as-is without any warranty either expressed or
%% implied; without even the implied warranty of MERCHANTABILITY or
%% FITNESS FOR A PARTICULAR PURPOSE! 
%% User assumes all risk.
%% In no event shall IEEE or any contributor to this code be liable for
%% any damages or losses, including, but not limited to, incidental,
%% consequential, or any other damages, resulting from the use or misuse
%% of any information contained here.
%%
%% All comments are the opinions of their respective authors and are not
%% necessarily endorsed by the IEEE.
%%
%% This work is distributed under the LaTeX Project Public License (LPPL)
%% ( http://www.latex-project.org/ ) version 1.3, and may be freely used,
%% distributed and modified. A copy of the LPPL, version 1.3, is included
%% in the base LaTeX documentation of all distributions of LaTeX released
%% 2003/12/01 or later.
%% Retain all contribution notices and credits.
%% ** Modified files should be clearly indicated as such, including  **
%% ** renaming them and changing author support contact information. **
%%
%% File list of work: IEEEtran.cls, IEEEtran_HOWTO.pdf, bare_adv.tex,
%%                    bare_conf.tex, bare_jrnl.tex, bare_jrnl_compsoc.tex
%%*************************************************************************

% *** Authors should verify (and, if needed, correct) their LaTeX system  ***
% *** with the testflow diagnostic prior to trusting their LaTeX platform ***
% *** with production work. IEEE's font choices can trigger bugs that do  ***
% *** not appear when using other class files.                            ***
% The testflow support page is at:
% http://www.michaelshell.org/tex/testflow/



% Note that the a4paper option is mainly intended so that authors in
% countries using A4 can easily print to A4 and see how their papers will
% look in print - the typesetting of the document will not typically be
% affected with changes in paper size (but the bottom and side margins will).
% Use the testflow package mentioned above to verify correct handling of
% both paper sizes by the user's LaTeX system.
%
% Also note that the "draftcls" or "draftclsnofoot", not "draft", option
% should be used if it is desired that the figures are to be displayed in
% draft mode.
%
\documentclass[conference]{IEEEtran}
% Add the compsoc option for Computer Society conferences.
%
% If IEEEtran.cls has not been installed into the LaTeX system files,
% manually specify the path to it like:
% \documentclass[conference]{../sty/IEEEtran}





% Some very useful LaTeX packages include:
% (uncomment the ones you want to load)


% *** MISC UTILITY PACKAGES ***
%
%\usepackage{ifpdf}
% Heiko Oberdiek's ifpdf.sty is very useful if you need conditional
% compilation based on whether the output is pdf or dvi.
% usage:
% \ifpdf
%   % pdf code
% \else
%   % dvi code
% \fi
% The latest version of ifpdf.sty can be obtained from:
% http://www.ctan.org/tex-archive/macros/latex/contrib/oberdiek/
% Also, note that IEEEtran.cls V1.7 and later provides a builtin
% \ifCLASSINFOpdf conditional that works the same way.
% When switching from latex to pdflatex and vice-versa, the compiler may
% have to be run twice to clear warning/error messages.






% *** CITATION PACKAGES ***
%
%\usepackage{cite}
% cite.sty was written by Donald Arseneau
% V1.6 and later of IEEEtran pre-defines the format of the cite.sty package
% \cite{} output to follow that of IEEE. Loading the cite package will
% result in citation numbers being automatically sorted and properly
% "compressed/ranged". e.g., [1], [9], [2], [7], [5], [6] without using
% cite.sty will become [1], [2], [5]--[7], [9] using cite.sty. cite.sty's
% \cite will automatically add leading space, if needed. Use cite.sty's
% noadjust option (cite.sty V3.8 and later) if you want to turn this off.
% cite.sty is already installed on most LaTeX systems. Be sure and use
% version 4.0 (2003-05-27) and later if using hyperref.sty. cite.sty does
% not currently provide for hyperlinked citations.
% The latest version can be obtained at:
% http://www.ctan.org/tex-archive/macros/latex/contrib/cite/
% The documentation is contained in the cite.sty file itself.






% *** GRAPHICS RELATED PACKAGES ***
%
\ifCLASSINFOpdf
  % \usepackage[pdftex]{graphicx}
  % declare the path(s) where your graphic files are
  % \graphicspath{{../pdf/}{../jpeg/}}
  % and their extensions so you won't have to specify these with
  % every instance of \includegraphics
  % \DeclareGraphicsExtensions{.pdf,.jpeg,.png}
\else
  % or other class option (dvipsone, dvipdf, if not using dvips). graphicx
  % will default to the driver specified in the system graphics.cfg if no
  % driver is specified.
  % \usepackage[dvips]{graphicx}
  % declare the path(s) where your graphic files are
  % \graphicspath{{../eps/}}
  % and their extensions so you won't have to specify these with
  % every instance of \includegraphics
  % \DeclareGraphicsExtensions{.eps}
\fi
% graphicx was written by David Carlisle and Sebastian Rahtz. It is
% required if you want graphics, photos, etc. graphicx.sty is already
% installed on most LaTeX systems. The latest version and documentation can
% be obtained at: 
% http://www.ctan.org/tex-archive/macros/latex/required/graphics/
% Another good source of documentation is "Using Imported Graphics in
% LaTeX2e" by Keith Reckdahl which can be found as epslatex.ps or
% epslatex.pdf at: http://www.ctan.org/tex-archive/info/
%
% latex, and pdflatex in dvi mode, support graphics in encapsulated
% postscript (.eps) format. pdflatex in pdf mode supports graphics
% in .pdf, .jpeg, .png and .mps (metapost) formats. Users should ensure
% that all non-photo figures use a vector format (.eps, .pdf, .mps) and
% not a bitmapped formats (.jpeg, .png). IEEE frowns on bitmapped formats
% which can result in "jaggedy"/blurry rendering of lines and letters as
% well as large increases in file sizes.
%
% You can find documentation about the pdfTeX application at:
% http://www.tug.org/applications/pdftex





% *** MATH PACKAGES ***
%
%\usepackage[cmex10]{amsmath}
% A popular package from the American Mathematical Society that provides
% many useful and powerful commands for dealing with mathematics. If using
% it, be sure to load this package with the cmex10 option to ensure that
% only type 1 fonts will utilized at all point sizes. Without this option,
% it is possible that some math symbols, particularly those within
% footnotes, will be rendered in bitmap form which will result in a
% document that can not be IEEE Xplore compliant!
%
% Also, note that the amsmath package sets \interdisplaylinepenalty to 10000
% thus preventing page breaks from occurring within multiline equations. Use:
%\interdisplaylinepenalty=2500
% after loading amsmath to restore such page breaks as IEEEtran.cls normally
% does. amsmath.sty is already installed on most LaTeX systems. The latest
% version and documentation can be obtained at:
% http://www.ctan.org/tex-archive/macros/latex/required/amslatex/math/





% *** SPECIALIZED LIST PACKAGES ***
%
%\usepackage{algorithmic}
% algorithmic.sty was written by Peter Williams and Rogerio Brito.
% This package provides an algorithmic environment fo describing algorithms.
% You can use the algorithmic environment in-text or within a figure
% environment to provide for a floating algorithm. Do NOT use the algorithm
% floating environment provided by algorithm.sty (by the same authors) or
% algorithm2e.sty (by Christophe Fiorio) as IEEE does not use dedicated
% algorithm float types and packages that provide these will not provide
% correct IEEE style captions. The latest version and documentation of
% algorithmic.sty can be obtained at:
% http://www.ctan.org/tex-archive/macros/latex/contrib/algorithms/
% There is also a support site at:
% http://algorithms.berlios.de/index.html
% Also of interest may be the (relatively newer and more customizable)
% algorithmicx.sty package by Szasz Janos:
% http://www.ctan.org/tex-archive/macros/latex/contrib/algorithmicx/




% *** ALIGNMENT PACKAGES ***
%
%\usepackage{array}
% Frank Mittelbach's and David Carlisle's array.sty patches and improves
% the standard LaTeX2e array and tabular environments to provide better
% appearance and additional user controls. As the default LaTeX2e table
% generation code is lacking to the point of almost being broken with
% respect to the quality of the end results, all users are strongly
% advised to use an enhanced (at the very least that provided by array.sty)
% set of table tools. array.sty is already installed on most systems. The
% latest version and documentation can be obtained at:
% http://www.ctan.org/tex-archive/macros/latex/required/tools/


%\usepackage{mdwmath}
%\usepackage{mdwtab}
% Also highly recommended is Mark Wooding's extremely powerful MDW tools,
% especially mdwmath.sty and mdwtab.sty which are used to format equations
% and tables, respectively. The MDWtools set is already installed on most
% LaTeX systems. The lastest version and documentation is available at:
% http://www.ctan.org/tex-archive/macros/latex/contrib/mdwtools/


% IEEEtran contains the IEEEeqnarray family of commands that can be used to
% generate multiline equations as well as matrices, tables, etc., of high
% quality.


%\usepackage{eqparbox}
% Also of notable interest is Scott Pakin's eqparbox package for creating
% (automatically sized) equal width boxes - aka "natural width parboxes".
% Available at:
% http://www.ctan.org/tex-archive/macros/latex/contrib/eqparbox/





% *** SUBFIGURE PACKAGES ***
%\usepackage[tight,footnotesize]{subfigure}
% subfigure.sty was written by Steven Douglas Cochran. This package makes it
% easy to put subfigures in your figures. e.g., "Figure 1a and 1b". For IEEE
% work, it is a good idea to load it with the tight package option to reduce
% the amount of white space around the subfigures. subfigure.sty is already
% installed on most LaTeX systems. The latest version and documentation can
% be obtained at:
% http://www.ctan.org/tex-archive/obsolete/macros/latex/contrib/subfigure/
% subfigure.sty has been superceeded by subfig.sty.



%\usepackage[caption=false]{caption}
%\usepackage[font=footnotesize]{subfig}
% subfig.sty, also written by Steven Douglas Cochran, is the modern
% replacement for subfigure.sty. However, subfig.sty requires and
% automatically loads Axel Sommerfeldt's caption.sty which will override
% IEEEtran.cls handling of captions and this will result in nonIEEE style
% figure/table captions. To prevent this problem, be sure and preload
% caption.sty with its "caption=false" package option. This is will preserve
% IEEEtran.cls handing of captions. Version 1.3 (2005/06/28) and later 
% (recommended due to many improvements over 1.2) of subfig.sty supports
% the caption=false option directly:
%\usepackage[caption=false,font=footnotesize]{subfig}
%
% The latest version and documentation can be obtained at:
% http://www.ctan.org/tex-archive/macros/latex/contrib/subfig/
% The latest version and documentation of caption.sty can be obtained at:
% http://www.ctan.org/tex-archive/macros/latex/contrib/caption/




% *** FLOAT PACKAGES ***
%
%\usepackage{fixltx2e}
% fixltx2e, the successor to the earlier fix2col.sty, was written by
% Frank Mittelbach and David Carlisle. This package corrects a few problems
% in the LaTeX2e kernel, the most notable of which is that in current
% LaTeX2e releases, the ordering of single and double column floats is not
% guaranteed to be preserved. Thus, an unpatched LaTeX2e can allow a
% single column figure to be placed prior to an earlier double column
% figure. The latest version and documentation can be found at:
% http://www.ctan.org/tex-archive/macros/latex/base/



%\usepackage{stfloats}
% stfloats.sty was written by Sigitas Tolusis. This package gives LaTeX2e
% the ability to do double column floats at the bottom of the page as well
% as the top. (e.g., "\begin{figure*}[!b]" is not normally possible in
% LaTeX2e). It also provides a command:
%\fnbelowfloat
% to enable the placement of footnotes below bottom floats (the standard
% LaTeX2e kernel puts them above bottom floats). This is an invasive package
% which rewrites many portions of the LaTeX2e float routines. It may not work
% with other packages that modify the LaTeX2e float routines. The latest
% version and documentation can be obtained at:
% http://www.ctan.org/tex-archive/macros/latex/contrib/sttools/
% Documentation is contained in the stfloats.sty comments as well as in the
% presfull.pdf file. Do not use the stfloats baselinefloat ability as IEEE
% does not allow \baselineskip to stretch. Authors submitting work to the
% IEEE should note that IEEE rarely uses double column equations and
% that authors should try to avoid such use. Do not be tempted to use the
% cuted.sty or midfloat.sty packages (also by Sigitas Tolusis) as IEEE does
% not format its papers in such ways.





% *** PDF, URL AND HYPERLINK PACKAGES ***
%
%\usepackage{url}
% url.sty was written by Donald Arseneau. It provides better support for
% handling and breaking URLs. url.sty is already installed on most LaTeX
% systems. The latest version can be obtained at:
% http://www.ctan.org/tex-archive/macros/latex/contrib/misc/
% Read the url.sty source comments for usage information. Basically,
% \url{my_url_here}.





% *** Do not adjust lengths that control margins, column widths, etc. ***
% *** Do not use packages that alter fonts (such as pslatex).         ***
% There should be no need to do such things with IEEEtran.cls V1.6 and later.
% (Unless specifically asked to do so by the journal or conference you plan
% to submit to, of course. )


% correct bad hyphenation here
\hyphenation{op-tical net-works semi-conduc-tor}


\begin{document}
%
% paper title
% can use linebreaks \\ within to get better formatting as desired
\title{Protecting Private Data in the Cloud: A Path Oblivious RAM Protocol}


% author names and affiliations
% use a multiple column layout for up to three different
% affiliations
\author{\IEEEauthorblockN{Ethan Zou}
\IEEEauthorblockA{Lexington High School\\ Lexington, MA \\
Email: zouethan19@gmail.com}
\and
\IEEEauthorblockN{Nathan Wolfe}
\IEEEauthorblockA{The Roxbury Latin School\\ Roxbury, MA\\
Email: nathanwolferl@gmail.com}}

% conference papers do not typically use \thanks and this command
% is locked out in conference mode. If really needed, such as for
% the acknowledgment of grants, issue a \IEEEoverridecommandlockouts
% after \documentclass

% for over three affiliations, or if they all won't fit within the width
% of the page, use this alternative format:
% 
%\author{\IEEEauthorblockN{Michael Shell\IEEEauthorrefmark{1},
%Homer Simpson\IEEEauthorrefmark{2},
%James Kirk\IEEEauthorrefmark{3}, 
%Montgomery Scott\IEEEauthorrefmark{3} and
%Eldon Tyrell\IEEEauthorrefmark{4}}
%\IEEEauthorblockA{\IEEEauthorrefmark{1}School of Electrical and Computer Engineering\\
%Georgia Institute of Technology,
%Atlanta, Georgia 30332--0250\\ Email: see http://www.michaelshell.org/contact.html}
%\IEEEauthorblockA{\IEEEauthorrefmark{2}Twentieth Century Fox, Springfield, USA\\
%Email: homer@thesimpsons.com}
%\IEEEauthorblockA{\IEEEauthorrefmark{3}Starfleet Academy, San Francisco, California 96678-2391\\
%Telephone: (800) 555--1212, Fax: (888) 555--1212}
%\IEEEauthorblockA{\IEEEauthorrefmark{4}Tyrell Inc., 123 Replicant Street, Los Angeles, California 90210--4321}}




% use for special paper notices
%\IEEEspecialpapernotice{(Invited Paper)}




% make the title area
\maketitle


\begin{abstract}
%\boldmath
We live in a world where our personal data are both valuable and vulnerable to misappropriation through exploitation of security vulnerabilities in online services. For instance, Dropbox, a popular cloud storage tool, has certain security flaws that can be exploited to compromise a user’s data, one of which being that a user's access pattern is unprotected. We have thus created an implementation of Path Oblivious RAM (Path ORAM) for Dropbox users to obfuscate path access information to patch this vulnerability. This implementation differs significantly from the standard usage of Path ORAM, in that we introduce several innovations, including a dynamically growing and shrinking tree architecture, multi-block fetching, block packing and the possibility for multi-client use.
\end{abstract}
% IEEEtran.cls defaults to using nonbold math in the Abstract.
% This preserves the distinction between vectors and scalars. However,
% if the conference you are submitting to favors bold math in the abstract,
% then you can use LaTeX's standard command \boldmath at the very start
% of the abstract to achieve this. Many IEEE journals/conferences frown on
% math in the abstract anyway.

% no keywords




% For peer review papers, you can put extra information on the cover
% page as needed:
% \ifCLASSOPTIONpeerreview
% \begin{center} \bfseries EDICS Category: 3-BBND \end{center}
% \fi
%
% For peerreview papers, this IEEEtran command inserts a page break and
% creates the second title. It will be ignored for other modes.
\IEEEpeerreviewmaketitle



\section{Introduction}
% no \IEEEPARstart
Nowadays, with many people possessing a multitude of devices, many of which have limited local storage, it is popular to store files in the cloud. Among the most popular cloud storage tools is Dropbox, which is compatible with computers and mobile devices, and allows users to store up to 2 GB of files for free (additional storage is available for a monthly fee).
For most users, Dropbox presents adequate security; all files are encrypted when stored on their servers. Nevertheless some will rightfully find certain issues with their system. One is that Dropbox can see which files you edit. This, known as the user’s access pattern, is a key piece of information, which Dropbox could potentially gather. Of course, Dropbox itself is trustworthy, but nowadays the US government has the power to request data from web services, and the access pattern may be included. While standard encryption offers some protection, it does nothing to hide a user’s access pattern.

Previous works have shown the importance in protecting access pattern. Memory access patterns can leak information such as a program’s control flow, which concerns conditional branches of the program and the order of commands executed \cite{HIDE}. Also, M.S. Islam et al. have shown how, from an encrypted email repository, adversaries can infer 80\% of search queries simply from access patterns \cite{Islam12}.

Oblivious RAM (ORAM) is designed specifically to obfuscate what files in a database are being edited. This obfuscation of the user’s access pattern provides a critical additional layer of security. Thus, in our research project we have implemented an ORAM system for use with Dropbox. Along the way, we have discovered various innovations and optimizations that are unique to the Dropbox usage scenario. First, we have designed a system where ORAM on Dropbox can be used between multiple user computers. We also propose the optimizations of multi-block fetching, which improves performance by 51.9\%, and block packing, which saves 50-70\% storage overhead. Finally, we propose a dynamically growing and shrinking tree, providing us with the necessary storage size flexibility to work with Dropbox.

In this paper, we will first provide some background on ORAM. We will then explain the system architecture behind our ORAM implementation. Then, after explaining our various optimizations and improvements to the base system, we will present some evaluations performed on the system to validate its effectiveness and assess its performance.


\section{Background}

\subsection{Oblivious RAM}
First, what is ORAM? Imagine that you have a table with papers lying on it in a common room at your office building. Occasionally you want to read or write on these papers. The problem is that your coworkers can see when you read or write on a sheet, when you really want to keep which one you are working on a secret. One good option to hide which you are working on is to take all of the papers into your private office, do what you want, and then bring them all back out to the table again. This is what we call naive Oblivious RAM (ORAM). Your coworkers are analogous to adversaries, the table outside is untrusted storage, and your own office is trusted. The sheets of paper represent blocks of data, and working on a sheet is analogous to accessing it, that is, reading or editing it. The point of the system we just described, and of any ORAM system, is to hide your access pattern: which sheet you are working on. This naive system works well to achieve this goal, but is obviously very inefficient. (Why even keep the papers outside if you bring them into your own room all the time anyway?)

In the past three decades, researchers have strived to improve the efficiency and reduce the trusted storage requirement of ORAM \cite{G87, GM11, O90, SCSL11, SSS12}. A breakthrough in implementation is achieved through Path ORAM \cite{PathORAM}. Path ORAM’s performance and trusted storage requirement are both logarithmic in its capacity. Furthermore, its algorithm is extremely simple and suitable for real world implementation. Immediately after its proposal in 2012, Path ORAM gained popularity in many applications due to its simplicity and efficiency. In this work, we adapted it for use with Dropbox.

\subsection{Path Oblvious RAM}
To make the Oblivious RAM system more efficient, researchers introduced Path ORAM. In this system, blocks of data are arranged into a binary tree, with a few blocks at each node of the tree. (A binary tree, pictured in {\bf Figure 2-1}, is a tree in which each node has up to two children.) Instead of accessing all blocks in order to keep a specific one secret, we can simply access all blocks along the target’s path. “Accessing” in this context means fetching a block from untrusted storage and placing it in memory to be worked on. Work on a specific block is still hidden because a group of blocks are accessed together, but the groups are much smaller, improving speed. 

diagram

We summarize how Path ORAM works here. The reader can refer to \cite{PathORAM} for more details. The tree is a binary tree data structure of height $L$ and $2^L$ leaves stored on the server as a one-dimensional array. The levels are numbered 0 to $L$, where level 0 is the root, level $L$ contains the leaves (in a complete tree), and levels in between contain non-leaf nodes. Each node in the binary tree is called a bucket, and contains $z$ blocks. If a bucket contains less than $z$ “real” blocks, the remainder are composed of dummy blocks, which are indistinguishable from real blocks because buckets are stored as encrypted fixed-size arrays.

On the client side, we construct a data structure called the {\bf stash}, which stores a small subset of the data blocks, and a {\bf position map} that assigns each block to a path in the tree. Every path is uniquely defined as leading from the root to one of the $2^L$ leaf buckets. Importantly, mapping is random, allowing for multiple unrelated blocks to be mapped to one leaf. This is shown in {\bf Figure 2-1}: for example, both blocks B and E are mapped to leaf 1, and are required to reside on the path defined by this leaf (referred to as path 1). The position map is updated as blocks are accessed and their leaf assignments changed. This is critical because Path ORAM maintains the invariant that if the block is not on its assigned path, it is in the stash (if it is not, there is a bug).

Reading/writing a block consists of several steps. First, we consult the position map to determine what path the target block is on. Then, we fetch the blocks on that path from the tree. Based on the invariant just mentioned, it should be among those fetched blocks or in the stash. We can then securely access the target block. After making any necessary changes to the block we assign it to a different path using the position map. Finally we take all blocks originally fetched and write them back into the tree.

In the past many papers have focused on Path ORAM systems and their optimizations. However, this is the first applying Path ORAM directly for use with Dropbox, and the first to explore optimizations unique to the Dropbox usage case.



\section{System Architecture}

We introduce the general organization of our design and several changes to the basic implementation to better suit Dropbox.

\subsection{Big Picture}

image

The overall interface of our design consists of three parts: the client, the Path ORAM implementation and Dropbox. Instead of the client directly writing and reading files to the Dropbox folder, we implemented the ORAM as an intermediary controller between the two. So a user will directly interact with the User File System (See Section 3.2), which in turn interacts with the ORAM controller to read and write files. The ORAM controller interacts with Dropbox by having the buckets of the tree written to the Dropbox folder. The Dropbox service will then sync the contents to the cloud. When reading a file, the bucket files are downloaded. As we will discuss in Section 3.3, a system with multiple computers accessing the same files requires the stash, position map, and other dictionaries to be written to Dropbox as well, which can be downloaded by other computers in the system. These structures which are normally kept in secure storage have to be encrypted with a secret key.

\subsection{User File System}

Dropbox’s baseline functions allow users to read and write files to a remote server in the cloud. Thus, we needed a way for our ORAM system to support the reading and writing of files of various sizes. Our solution was implementing the User File System, which enables the writing of files of different sizes. 

The User File System interacts with the ORAM controller and provides the users with an interface to read, update and delete files of arbitrary size. The User File System consists of several dictionaries which hold vital metadata. Its main feature is to partition files into chunks, whose size is a parameter we call {\bf segment size}. These chunks are then each written to or read from the ORAM.

image

To write a file, we split it up into subfiles each containing a number of bytes equal to (or less than, in the case of the last subfile) the segment size. We then assign each subfile a unique segment ID (stored in a dictionary) that allows us to be able to track the individual segments when we want to read the file again. Then, we write each file segment to the ORAM and update another dictionary with the number of partitioned pieces in that particular file. When reading a file, we look up the file name in a dictionary to retrieve the number of segments in that file. Then, we obtain the individual segment IDs of each data segment and fetch them from the ORAM. The data is then stitched together and returned to the user. Deleting a file works similarly to reading. After looking up the segment IDs of the file fragments, we simply delete those data from the ORAM and delete their corresponding information from the two metadata dictionaries.

\subsection{Multi-Computer}

One of the major reasons cloud storage programs such as Dropbox are so popular is that a user can access and change files on one computer, and then move to another computer to access the same files. In order to support this feature, we added functionalities to the file systems.

When a user “logs off” from the computer, the dictionaries in the User File System, the position map, and the stash are written to the Dropbox folder, and then synced to the remote server. An important implication is that these files are encrypted and padded to a constant file size when written to Dropbox to ensure security and prevent an adversary from gaining information from file size or access pattern. When the user wants to access his files from another computer, he/she downloads these files from the server to reconstruct the User File System and ORAM metadata structures.

\subsection{Initial Evaluations}

Our current implementation is significantly slower than directly using Dropbox without ORAM. Thus, the implementation has some apparent drawbacks. The major problem comes from partitioning an immensely large file into millions of segments (based on the assigned segment size), which means that reading or writing one large file takes a very long time. Along with this, a static tree size is a problem when using Dropbox because we don’t want space to be wasted (more space equals more money). With these issues in mind, we sought to develop optimizations to improve upon our initial implementation.


\section{Optimizations}

In this section we provide detailed explanations and motivation for the new optimizations we have implemented: multi-block fetching, block packing and a dynamically growing and shrinking tree size.

\subsection{Multi-Block Fetching}

One idea to improve performance dramatically is to fetch multiple data blocks in each access, thus cutting down on the total number of accesses required. Unfortunately, usually when using ORAM, it is unclear how data blocks are grouped outside, so it is hard to know which groups of blocks to fetch together. However, in our implementation of ORAM with Dropbox, the data segments stored can be grouped in a defined way, with each grouping corresponding to a particular file. Thus we can cut down on the number of tree accesses we perform by putting segments from the same file on the same path in the tree and fetching them at the same time. The technique is as simple as grouping segments from each file into n-tuples, mapping them all to the same path, and then fetching them all at the same time. Theoretically this should reduce the number of accesses needed by a factor of n. Unfortunately, however, grouping a great number of segments together in the same path produces inefficiency. Any individual path may become congested, slowing down the operation. Through evaluations of file access speed we have managed to find an optimal value for n for this grouping process.

\subsection{Block Packing}

One way to speed up the ORAM operations is to cut down on the number of partitions of a file. In order to do that, we need to use a large segment size. However, larger segment sizes hurt small files because a tremendous amount of space is wasted, and it takes longer to read small files. 

With this in mind, our approach to improving this aspect of the ORAM was to implement block packing, which essentially packs more than one file segment together into one block. The motivation behind this approach comes from the idea that a small file does not necessarily need to be the only file present in a block. Additionally, leftover data that does not divide evenly into the segment size can also be packed into the empty spaces of a block. By doing this, space is saved and not wasted, resulting in a smaller tree and smaller path lengths (see Section 4.4), and consequently reduces file access time (though only slightly).

Block packing works as follows. We add two more dictionaries to the User File System: one holds the amount of space left in a block and the other holds the start and end offsets of a certain file segment inside of a block. We write a file of an arbitrary size. The “full” segments of data (take up an entire block) are written normally to the ORAM. We then look for available space in a block for the very last segment (which, in the case of a small file will be the entire file). If we find a block that already has some data in it and that has sufficient excess space, we append the final segment to the end of the already existing data, and update the dictionaries accordingly. If we do not find an available block, we write the data to an empty block and add the ID of that block to the dictionary with its amount of empty space. When we want to read the file, all the segments except the last are read normally. We find the position of the last segment in the block and read out the specific portion of the data.

Deleting a file becomes slightly more complicated. All file segments except the last one are deleted normally. When deleting the last segment of a file, we run into the problem of there possibly being a “hole” of no data inside a block. Our solution was to shift the data after the deleted segment as far up in the block as possible so that there are no gaps in the middle of a block. We then update the dictionaries for the shifted segments of data. 

\subsection{Dynamic Growing and Shrinking Tree}

A normal ORAM implementation involves a storage space that is fixed in size (e.g., main memory or hard drive of a computer). However, Dropbox presents the unique problem of a dynamically sized storage space. First, users may wish to store sensitive files in an ORAM system alongside less important files stored unencrypted in Dropbox. By supporting resizing of the tree we support this usage scenario. Second, Dropbox allows users to increase the size of their storage by either paying or sharing the service with friends. By supporting tree resizing we allow users to change their storage space without having to create a new ORAM library each time. Thus, resizing provides convenience to the user in a way unique to this project.

The first problem we encounter is to determine when to resize. Previous work suggested that the optimal utilization (meaning the fraction data block slots that are used) is around 50\% \cite{oram-isca13}.

Our implementation supports both growing and shrinking. Both functions are triggered based on current utilization,. When the utilization goes above a certain threshold we grow the tree until the utilization goes back down to a target level. When it goes below a certain threshold we shrink (to save Dropbox space), with the utilization going back up to a target level.

Growing the tree works as follows. First, we add the correct number of buckets to the leaf end of the tree. Obviously this reveals the information that we are growing, and that the total size of our files stored has gone beyond some threshold, but that information is both non-critical and practically unavoidable to leak. We then correct all leaves in both the position map and the stash in order to conform to the new leaf numbers. Nodes with new children are invalidated as leaves, so records pointing to these nodes are reassigned to one of the children.

Shrinking works similarly; first we remove the correct number of buckets from the leaf end of the tree, dumping their contents into the stash. Again, this leaks the information that our total file size must have decreased. We then correct all leaves in the position map and the stash by truncating the path so that it lines up with the correct leaf, and leave the ones in the tree to sort out later.

\section{Evaluations}
In this section, we evaluate the performance of our optimizations. 
\subsection{Methodology}

The CPU Model of the machine used for the experiments is Intel(R) Core\textsuperscript{TM} i3-2100 CPU @ 3.10GHz. The disk model is ST31000524AS (7200 RPM). Windows 7 operating system and Python 3.4 were used. Our encryption algorithm was Advanced Encryption Standard (AES) 128-bit in counter (CTR) mode, in the PyCrypto package v2.6.1.

For all experiments, the parameters were kept at constant values (unless we were testing that parameter). The ORAM tree size (if auto resizing is off) was dependent on the amount of data written to it, twice the total file size. Segment size was 64 KB (optimal as indicated by Figure 5-2) with bucket sizes of z = 3.  The maximum stash size was 100. The utilization ratio that triggers tree growth was 0.45 and for tree shrinkage 0.55. We use throughput as the performance metric, which is a measurement of how fast files can be read from and written to our ORAM-enabled Dropbox. It is expressed in bits per second or multiples of it. For our experiments we use megabytes per second (MB/s).

For all the experiments except the one in Section 5.5, we utilize a distribution of files of various sizes to better simulate the contents/actions of a “real” file system, by reading/writing 1000 files based upon this cumulative distribution. This file distribution was chosen from previous work done on file system workloads \cite{roselli2000comparison}, and is reproduced below in Figure 5-1.

graph

\subsection{Optimal Segment Size}
graph

We performed a sweep study to evaluate the optimal segment size into which files are partitioned for storage. The sweep study shows that optimal segment size is 64 KB. It is not surprising that there is an optimal segment size. For small segment sizes, large files are handled inefficiently because they gets partitioned into many more pieces, resulting in a massive number of operations to read/write one file. For large segment sizes, small files take longer to read because of the majority of filler data in a block.

\subsection{Initial ORAM Efficiency}
graph

We evaluated the overhead cost of implementing of Path ORAM with that of encryption alone. As a baseline we compared this to reading/writing files to the local disk.  Encryption added a little overhead (about 1.7 times slower) to the operations. Incorporating ORAM made the operations an additional 37 times slower when compared to encryption alone. As indicated by the table, the performance of our Dropbox-ORAM implementation without any optimizations is very slow, as expected. The remainder of this section evaluates the effectiveness of our improvements.

\subsection{Multi-Block Fetching}
graph

As noted in Section 4.1, we utilized multi-block fetching with segment grouping to improve the efficiency of our tree architecture. From the graph we can see that the optimal grouping size appears to be 3, which also indicates that multiple block fetching does improve performance. However, larger grouping sizes create more congestion in some paths. To compensate for congestion, a technique called background eviction has to be employed, which takes additional overhead \cite{oram-isca13, Yu14DSB}.

\subsection{Block Packing}
two graphs

To evaluate the impact of block packing on performance, we assessed two dependent parameters, tree size and data throughput. For both tests, we wrote 1000 files of the same size to the ORAM and then recorded the throughput of reading the files 1000 times. Automatic resizing was turned on in order to illustrate the space saving aspect of this optimization. The file sizes of 16, 32 and 70 KB were chosen specifically for this test. Without block packing, the size of the tree was expected to be the same for both the 16 KB files and the 32 KB files because each file will take up one block, and with block packing, the size was expected to be $1/4$ and $1/2$ of the previous sizes because 4 times and 2 times as many files can be packed together in a block, respectively. The results confirm our expectations. The 70 KB file was chosen to illustrate that block packing works with the leftover segments. As shown in the graph, the tree was about two times smaller, because the extra 6 KB file segments were packed together, saving space. Block packing also yielded a small improvement in data throughput. The saved space results in a smaller tree structure. As a result the time taken for a path read/write is expected to be somewhat shorter.

\subsection{Dynamic Growing and Shrinking}
graph

To demonstrate the implementation of our dynamic tree structure, we tested the impact of random file addition or deletion to the tree structure. Thirty-two actions were performed using files of random sizes for this demonstration. Figure 5-6 displays the impact of an increasing and decreasing total file size stored in the tree. Whenever the tree utilization breaks the threshold set at $0.45$ or $0.55$, the tree to resizes to fulfill the target 50\% utilization.

\subsection{Total Optimization Impact}

All of our improvements taken together yield significant savings in both storage space and time. Block packing alone yielded a performance gain on average of 15\%, and multi-block fetching with groups of 3 yielded a performance gain of 52\%. With both optimizations running together we would expect total gain of about 60\%. The estimated throughput is 1.06 MB/s. 

\section{Future Work}

In the future we plan to more fully detail the merits and deficits of our Path ORAM implementation through additional tests and, where needed, further optimizations. We believe that this will allow us to grow our project into a full-fledged software package.

One additional optimization is to implement a “hybrid” ORAM system. This would combine the current Path system when using small files with a scan system, which is much more efficient for large files. We have also determined ways to optimize the stash algorithm, which we could implement in our system as well.

Our final goal for this project is to make it into a public software package. This would require several additions, including a better user interface and crash recovery solutions. Crash recovery poses a major obstacle to implementation; without a recovery mechanism, a crash in the middle of syncing with Dropbox could result in lost data.


% An example of a floating figure using the graphicx package.
% Note that \label must occur AFTER (or within) \caption.
% For figures, \caption should occur after the \includegraphics.
% Note that IEEEtran v1.7 and later has special internal code that
% is designed to preserve the operation of \label within \caption
% even when the captionsoff option is in effect. However, because
% of issues like this, it may be the safest practice to put all your
% \label just after \caption rather than within \caption{}.
%
% Reminder: the "draftcls" or "draftclsnofoot", not "draft", class
% option should be used if it is desired that the figures are to be
% displayed while in draft mode.
%
%\begin{figure}[!t]
%\centering
%\includegraphics[width=2.5in]{myfigure}
% where an .eps filename suffix will be assumed under latex, 
% and a .pdf suffix will be assumed for pdflatex; or what has been declared
% via \DeclareGraphicsExtensions.
%\caption{Simulation Results}
%\label{fig_sim}
%\end{figure}

% Note that IEEE typically puts floats only at the top, even when this
% results in a large percentage of a column being occupied by floats.


% An example of a double column floating figure using two subfigures.
% (The subfig.sty package must be loaded for this to work.)
% The subfigure \label commands are set within each subfloat command, the
% \label for the overall figure must come after \caption.
% \hfil must be used as a separator to get equal spacing.
% The subfigure.sty package works much the same way, except \subfigure is
% used instead of \subfloat.
%
%\begin{figure*}[!t]
%\centerline{\subfloat[Case I]\includegraphics[width=2.5in]{subfigcase1}%
%\label{fig_first_case}}
%\hfil
%\subfloat[Case II]{\includegraphics[width=2.5in]{subfigcase2}%
%\label{fig_second_case}}}
%\caption{Simulation results}
%\label{fig_sim}
%\end{figure*}
%
% Note that often IEEE papers with subfigures do not employ subfigure
% captions (using the optional argument to \subfloat), but instead will
% reference/describe all of them (a), (b), etc., within the main caption.


% An example of a floating table. Note that, for IEEE style tables, the 
% \caption command should come BEFORE the table. Table text will default to
% \footnotesize as IEEE normally uses this smaller font for tables.
% The \label must come after \caption as always.
%
%\begin{table}[!t]
%% increase table row spacing, adjust to taste
%\renewcommand{\arraystretch}{1.3}
% if using array.sty, it might be a good idea to tweak the value of
% \extrarowheight as needed to properly center the text within the cells
%\caption{An Example of a Table}
%\label{table_example}
%\centering
%% Some packages, such as MDW tools, offer better commands for making tables
%% than the plain LaTeX2e tabular which is used here.
%\begin{tabular}{|c||c|}
%\hline
%One & Two\\
%\hline
%Three & Four\\
%\hline
%\end{tabular}
%\end{table}


% Note that IEEE does not put floats in the very first column - or typically
% anywhere on the first page for that matter. Also, in-text middle ("here")
% positioning is not used. Most IEEE journals/conferences use top floats
% exclusively. Note that, LaTeX2e, unlike IEEE journals/conferences, places
% footnotes above bottom floats. This can be corrected via the \fnbelowfloat
% command of the stfloats package.



\section{Conclusion}
Using an ORAM system with Dropbox allows the user to hide a key piece of information, his/her access pattern. Doing so, however, costs a significant overhead. To counteract this, we have introduced a few optimizations, which, as shown in the evaluations section, cut down on the overhead and increase throughput. Additionally, to conform best with Dropbox, we have altered the Path ORAM system with improvements, including potential for multi-computer use and a dynamically resizing tree. These developments are unique to any Path ORAM work that has been previously described. Our goal is that continued optimization will increase Path ORAM’s speed sufficiently that it may be practical for mainstream use. 

To summarize our work, we first designed and implemented a Path ORAM system to work with Dropbox. We then performed optimizations including multi-block fetching, block packing, and a growing and shrinking tree, improving performance by an estimated 60\%, and reducing storage overhead by 50-70\%. Both of these savings provide us with flexibility that is necessary for Path ORAM to work harmoniously with Dropbox.





% conference papers do not normally have an appendix


% use section* for acknowledgement
\section*{Acknowledgment}


(Third person???) The authors would like to thank mentors Ling Ren and Xiangyao Yu for their guidance and insight, and professor Srini Devadas for suggesting the project and encouraging them along the way. They would also like to thank the MIT PRIMES program for making this research possible.





% trigger a \newpage just before the given reference
% number - used to balance the columns on the last page
% adjust value as needed - may need to be readjusted if
% the document is modified later
%\IEEEtriggeratref{8}
% The "triggered" command can be changed if desired:
%\IEEEtriggercmd{\enlargethispage{-5in}}

% references section

% can use a bibliography generated by BibTeX as a .bbl file
% BibTeX documentation can be easily obtained at:
% http://www.ctan.org/tex-archive/biblio/bibtex/contrib/doc/
% The IEEEtran BibTeX style support page is at:
% http://www.michaelshell.org/tex/ieeetran/bibtex/
%\bibliographystyle{IEEEtran}
% argument is your BibTeX string definitions and bibliography database(s)
%\bibliography{IEEEabrv,../bib/paper}
%
% <OR> manually copy in the resultant .bbl file
% set second argument of \begin to the number of references
% (used to reserve space for the reference number labels box)
% \begin{thebibliography}{1}

%\bibitem{IEEEhowto:kopka}
%H.~Kopka and P.~W. Daly, \emph{A Guide to \LaTeX}, 3rd~ed.\hskip 1em 
%plus 0.5em minus 0.4em\relax Harlow, England: Addison-Wesley, 1999.

%\end{thebibliography}

\bibliographystyle{IEEEtran}
\bibliography{paper}

% that's all folks
\end{document}


